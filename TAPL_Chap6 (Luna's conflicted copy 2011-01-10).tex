\documentclass[13pt,dvipdfm]{beamer}
\usepackage{amsmath,amssymb,amsthm}
\usepackage[all]{xy}


\usepackage{beamerthemeshadow}
\usefonttheme{professionalfonts}

\title{TAPL Chap6\\Nameless Representation of Terms}
\author{Kinebuchi Tomohiko}
\date{2011/01/09}

\begin{document}
\frame{\titlepage}

\frame
{
  \frametitle{まえがき}
  第 5 章で「束縛変数の名前付け替え」($\alpha$ 変換) を扱った.\\
  この方法は人間に分かりやすいし, 証明もしやすい.\\
  しかし実装には向かない.\\
  できれば一意な形になって欲しい.
}

\frame
{
  \frametitle{代入の扱い方}
  \begin{enumerate}
  \item 変数を記号で扱って, 名前付け替えで対応 (第 5 章での方法)
  \item 変数を記号で扱うが, 束縛変数は他のどの変数とも名前がかぶらない. ただし代入で不安定. (Barendregt convention)
  \item ある一意の形で項を表す. (この章で扱う方法)
  \item explicit substitutions
  \item 変数を使わない方法を取る. Combinatory logic
  \end{enumerate}
  どれを選ぶかは好き好き.
}

\frame
{
  \frametitle{代入の扱い方}
この本では 3 番目の de Bruijn による方法を選ぶ.\\
この方法で実装すると複雑な例にも対応できるし, 間違えたときにすぐ分かる.
}

\frame
{
  \frametitle{6.1 Terms and Contexts}
  de Bruijn のアイディアは読みづらいけど直接的.\\
  変数を数字で書いてしまう.\\
  これを de Bruijn index と呼ぶ.\\
  コンパイラのを書く人は同じ概念を ``static distance'' と呼んだりする.
}

\frame
{
  \frametitle{6.1.1 Exercise}
  やってみましょう. 端から順に当ててきます.\\
  plus に誤植があります.\\
  (n z s) の部分は (n s z) のはず.
}

\frame
{
  \frametitle{6.1.2 Definition}
  
  \begin{definition}
  $\mathcal{T} = \{\mathcal{T}_0, \mathcal{T}_1, \mathcal{T}_2,\dots\}$
  \begin{enumerate}
  \item $\mathsf{k} \in \mathcal{T}_n \Leftrightarrow 0 \leq \mathsf{k} < n$
  \item $\mathsf{t}_1 \in \mathcal{T}_n\ (n > 0) \Rightarrow \lambda.\mathsf{t}_1 \in \mathcal{T}_{n-1}$
  \item $\mathsf{t}_1 \in \mathcal{T}_n, \mathsf{t}_2 \in \mathcal{T}_n \Rightarrow (\mathsf{t}_1 \mathsf{t}_2) \in \mathcal{T}_n$
  \end{enumerate}
  \end{definition}

  5.3.1 (P.69) のような形式で定義する.
  5.3.1 と違うのは自由変数の個数をきっちり見てるところ.\\
  $\mathcal{T}_n$ (n-terms) =「多くとも n 個しか自由変数が無い項」
  t が閉項 (自由変数が 0 個) なら全ての $\mathcal{T}_n$ に属する.
}

\frame
{
  \frametitle{6.1.2}
  閉項の場合 de Bruijn 表現は一意に決まる.\\
  もちろん名前の付け替えで等しい項は de Bruijn 表現で同じ表現になる.\\
  自由変数がある場合を扱うために naming context を用意する.\\
  naming context = あらかじめ自由変数用の文字を用意して, その使う順序を決めたもの.\\
  \begin{align*}
    \Gamma = \mathsf{x} \mapsto 4 \\
    \mathsf{y} \mapsto 3 \\
    \mathsf{z} \mapsto 2 \\
    \mathsf{a} \mapsto 1 \\
    \mathsf{b} \mapsto 0
  \end{align*}

  重なるとまずいので登場する束縛変数のの数だけずらして使う.
}

\frame
{
  \frametitle{例}
  $\lambda \mathsf{x}.\ \mathsf{x}$ は $\lambda.\ 0$\\
  $\mathsf{x}\ (\mathsf{y}\ \mathsf{z})$ は $4\ (3\ 2)$\\
  $\lambda \mathsf{w}.\ \mathsf{y}\ \mathsf{w}$ は途中まで変換すると $\lambda.\ \mathsf{y}\ 0$\\
  ここで元々 $\mathsf{w}$ だった 0 と重ならないように context をずらす.
  \begin{align*}
    \Gamma = \mathsf{x} \mapsto 5 \\
    \mathsf{y} \mapsto 4 \\
    \mathsf{z} \mapsto 3 \\
    \mathsf{a} \mapsto 2 \\
    \mathsf{b} \mapsto 1
  \end{align*}
  なので $\lambda.\ 4\ 0$\\
  $\lambda \mathsf{w}.\ \lambda \mathsf{a}.\ \mathsf{x}$ は 2 つずれて $\lambda.\ \lambda.\ 6$
}

\frame
{
  \frametitle{6.1.3 Definition}
  いちいち書いてると面倒なので naming context を
  $\Gamma = \mathsf{x}_n, \mathsf{x}_{n-1},\dots,\mathsf{x}_0$
  と書いて, $x_i \mapsto i$ という対応付けを表すとする.\\
  $dom(\Gamma) \equiv \{\mathsf{x}_n, \mathsf{x}_{n-1},\dots,\mathsf{x}_0\}$
}

\frame
{
  \frametitle{context をずらす操作についてもう少し}
   \label{frame:context_shift}
  $\lambda w.\ \lambda a.\ x$ の 1 番外側では context は $\Gamma = x, y, z, a, b$.\\
  もちろん, これらは全て自由変数.\\
  ↓\\
  abstraction の 1 つ内側 ($\lambda a.\ x$ の部分) に入るとき $w$ と衝突しないようにずらさないといけない.\\
  新しい context $\Gamma' = x, y, z, a, b, w$ になると考えてもよい.\\
  ↓\\
  もう一度 abstraction の内側 ($x$ の部分) に入るとき $a$ と衝突しないようにずらさないといけないので,\\
  新しい context $\Gamma' = x, y, z, a, b, w, a$ になる.
}

\frame
{
  \frametitle{6.1.4 Exercise}
  さっきの $\mathcal{T} = \{\mathcal{T}_0, \mathcal{T}_1, \mathcal{T}_2,\dots\}$
  を 3.2.3 (P. 27) みたいに定義し直して, 最初の定義と同等であることを示せ.

  以下のように構成すれば良い.
  \begin{align*}
    \mathcal{S}_n^{i+1} &= \{0, 1, 2,..., n-1\} \\
    &\cup \{\lambda.t | t \in \mathcal{S}_{n+1}^i \} \\
    &\cup \{t_1 t_2 | t_1, t_2 \in \mathcal{S}_n^i\} \\
    \mathcal{S}_n = \bigcup_i \mathcal{S}_n^i
  \end{align*}
 ちなみにこの定義から $\mathcal{S}_n \subset \mathcal{S}_{n+1}$ が分かる.
}

\frame
{
  \xymatrix {
    \mathcal{S}_0^0 \ar[d]_{\text{app}} & \mathcal{S}_1^0 \ar[ld]_{\text{abs}} \ar[d] & \mathcal{S}_2^0 \ar[ld] \ar[d] & \mathcal{S}_3^0 \ar[ld] \ar[d] & \mathcal{S}_4^0 \ar[ld] & \dots \\
    \mathcal{S}_0^1 \ar[d]_{\text{app}} & \mathcal{S}_1^1 \ar[ld]_{\text{abs}} \ar[d] & \mathcal{S}_2^1 \ar[ld] \ar[d] & \mathcal{S}_3^1 \ar[ld] & \dots \\
    \mathcal{S}_0^2 & \mathcal{S}_1^2 & \mathcal{S}_2^2 & \dots \\
    \dots & \dots & \dots \\
    \dots & \mathcal{S}_n^i \ar[d]_{\text{app}} & \mathcal{S}_{n+1}^i \ar[ld]_{\text{abs}} \ar[d] & \mathcal{S}_{n+2}^i \ar[ld] & \dots \\
    \dots & \mathcal{S}_n^{i+1} & \mathcal{S}_{n+1}^{i+1} & \dots
  }
}

\frame
{
  \frametitle{6.1.5 Exercise}
  通常の項の表現から de Bruijn 表現へ変換する関数 $removenames_\Gamma (\mathsf{t})$ を書け.
  $FV(\mathsf{t}) \subset dom(\Gamma)$

  de Bruijn 表現から通常の項の表現へ変換する関数 $restorenames_\Gamma (\mathsf{t})$ を書け.\\
  当然この 2 つは逆関数の関係にないといけない. \\
  基本のアイディアは \ref{frame:context_shift} シート目の内容
}

\frame
{
  \frametitle{巻末の解答について}
  application の場合は context は変化しないと思う.\\
  変化してしまうと 6.2.1 の定義と食い違うし, $x$ がどこから出てきたか分からない. 誤植? \\
  以下が正しいと思う.\\
  $removenames_\Gamma (t_1\ t_2) = removenames_\Gamma (t_1)\ removenames_\Gamma (t_2)$
}

\frame
{
  \frametitle{6.2 Shifting and Substitution}
  代入を扱う前にシフト操作を定義する.\\
  \ref{frame:context_shift} シート目の内容を厳密に定義する.
}

\frame
{
  \frametitle{6.2.1 Definition}
  これを使うと自由変数だけシフトすることができる.
  \begin{definition}
    \begin{align*}
      \uparrow^d_c (\mathsf{k}) &=
      \begin{cases}
        \mathsf{k} & \text{if}\ \mathsf{k} < c \\
        \mathsf{k} + d & \text{if}\ \mathsf{k} \geq c
      \end{cases} \\
      \uparrow^d_c (\lambda.\ \mathsf{t}_1) &= \lambda.\ \uparrow^d_{c+1} (\mathsf{t}_1) \\
      \uparrow^d_c (\mathsf{t}_1\ \mathsf{t}_2) &= \uparrow^d_c (\mathsf{t}_1)\  \uparrow^d_c(\mathsf{t}_2)
    \end{align*}
  \end{definition}
}

\frame
{
  \frametitle{6.2.2 Exercise}
  計算してみる.
}

\frame
{
  \frametitle{6.2.3 Exercise}
  項のサイズについての帰納法で証明してみる.
}

\frame
{
  \frametitle{6.2.4 Definition}
  代入が書けるようになった. \\

  \begin{definition}
    \begin{align*}
      [\mathsf{j} \mapsto \mathsf{s}]\mathsf{k} &=
      \begin{cases}
        s & \text{if}\ \mathsf{k} = \mathsf{j} \\
        k & \text{otherwise}
      \end{cases} \\
      [\mathsf{j} \mapsto \mathsf{s}](\lambda.\ \mathsf{t}_1) &=
      \lambda.\ [\mathsf{j+1} \mapsto \uparrow^{1} (\mathsf{s})]\mathsf{t}_1 \\
      [\mathsf{j} \mapsto \mathsf{s}](\mathsf{t}_1\ \mathsf{t}_2) &=
      [\mathsf{j} \mapsto \mathsf{s}]\mathsf{t}_1\ [\mathsf{j} \mapsto \mathsf{s}]\mathsf{t}_2
    \end{align*}
  \end{definition}
}

\frame
{
  \frametitle{6.2.5 Exercise}

  4 では代入した後の自由変数の index の調整が必要
}

\frame
{
  \frametitle{6.2.6 Exercise}
  6.2.3 の結果を使う.
}

\frame
{
  \frametitle{6.2.7 Exercise}
  定義をそらで書けるかのチェック.
  各自やってください.
}

\frame
{
  \frametitle{6.2.8 Exercise}
  「nameless term を使った代入操作と ordinary term を使った代入操作が一致する」
  という主張を言うのに必要な定理を書き, 証明する.

  6.1.5 で $removenames_{\Gamma}$ と $restorenames_{\Gamma}$ は作ったので \\
  $restorenames_{\Gamma}([\mathsf{j} \mapsto \mathsf{s}] removenames(\lambda \mathsf{x}. \mathsf{t}_1)) = [\mathsf{j} \mapsto \mathsf{s}] (\lambda \mathsf{x}. \mathsf{t}_1)$
  を示せば良い. (残りは楽なので省略)
}

\frame
{
  \frametitle{6.3 Evaluation}
  代入までできたので, 残るは $\beta$ 簡約の定義.\\
  もちろん代入を使う.\\
  ポイントとなるのは代入を行うと束縛変数が 1 つ消える.\\
  その空いた分を詰めなくてはいけない.\\
  (6.2.5 の 4 を解くと分かる.)\\
  その再調整操作を入れて\\
  $(\lambda.\ t_{12})\ \mathsf{v}_2 \longrightarrow \uparrow^{-1} ([0 \mapsto \uparrow^{1} (\mathsf{v}_2)]\ t_{12})$\\
  こんなルールになる.
}

\frame
{
  \frametitle{6.3.1 Exercise}
  -1 のシフトがあるけど大丈夫か?\\
  項 $([0 \mapsto \uparrow^{1} (\mathsf{v}_2)]\ t_{12})$ に 0 は出てこないので -1 シフトしても大丈夫
}

\frame
{
  \frametitle{6.3.2 Exercise}
  de Bruijn level (reversed de Bruijn repr.)
  naming context $\Gamma = x_0, x_2,\dots,x_n$
}



\end{document}
